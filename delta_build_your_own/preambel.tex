% 1. Definieren der Dokumentklasse.
\documentclass[
  pdftex,                           % PDFTex verwenden um ausschliesslich ein PDF zu erzeugen
  a4paper,                          % A4 Papier
  doubleside,                       % Doppelseitiger Druck
  11pt,                             % Standard Schriftgrösse
  parskip=half,                     % Europäischer Satz mit Abstand zwischen Absätzen
  headsepline,                      % Linie nach Kopfzeile
  ]{scrreprt}                       % KOMA-Script Klasse

% 2. Zeichencodierung und Zeichensatz
\usepackage[utf8]{inputenc}         % 'utf8' -> nicht kompatibel mit TeXnicCenter (unterstützt kein UTF-8)
\usepackage[T1]{fontenc}            % 'T1' Codierung für die Schrift

% 3. Lokalisierung
\usepackage[english, american]{babel}
\usepackage{textcomp}               % Euro Symbol (\texteuro}
\usepackage{upgreek}				% Paket für nichtkursive griechische Kleinbuchstaben

% 4. Schriften
\usepackage{courier}                % Courier laden (wird für Code verwendet)
\usepackage{helvet}                 % Helvetica laden

% 5. Textpakete und Optionen
\usepackage{setspace}               % Paket für Textabstände
\onehalfspacing                     % 1.5-facher Zeilenabstand
\usepackage[section]{placeins}      % Paket für Flusskontrolle -> verhindert, das Flussobjekte aus
                                    % einer Section heraus geschoben werden
\usepackage{enumerate}

% 6. Seiteneinstellungen
\topmargin-12mm                     % Oberer Seitenrand verstellen -> Hack, verbessern
\headheight0.95cm                   % Höhe der Kopfzeile
\textheight24.2cm                   % Texthöhe (ohne Kopf- und Fusszeile)
\footskip1.2cm                      % Abstand zwischen Text und unterem Ende der Fusszeile
\raggedbottom                       % Kein vertikaler Blocksatz

% 7. Kopf-/Fusszeilen, Fussnoten
\usepackage{scrpage2}
\usepackage{remreset}
\makeatletter
\@removefromreset{footnote}{chapter}
\makeatother

% 8. Bilder, Tabellen, Boxen
\usepackage[pdftex]{graphicx}       % Bilder
\usepackage{booktabs}               % mehrere Befehle für die Tabellen
\usepackage{array}
\usepackage{framed}                 % Paket für Boxen
\usepackage{pdfpages}
\graphicspath{                      % Suchpfade für Bilder
  {img/},
  {Bilder/},
  {Grafiken/},
  {images/},
  {template/},
  {Img/},
  {abb/},
  {}}
\usepackage[format=hang,justification=centering]{caption}   
\usepackage{csvsimple}

% 9. Farben
\usepackage{color}                  % Paket für Farben laden
\definecolor{LinkColor}{rgb}{0,0,0} % Verschiedene Farbdefinitionen
\definecolor{darkblue}{rgb}{0,0,0.6}
\definecolor{darkred}{rgb}{0.6,0,0}
\definecolor{darkgreen}{rgb}{0,0.6,0}
\definecolor{red}{rgb}{0.98,0,0}
\definecolor{lstbggray}{rgb}{0.95,0.95,0.95}
\definecolor{gray50}{rgb}{0.5,0.5,0.5}

\definecolor{ntblightgray}{gray}{0.8}
\definecolor{ntbdarkgray}{gray}{0.7}
\definecolor{ntbblue}{cmyk}{1, 0.45, 0, 0.18}
\definecolor{ntblightblue}{cmyk}{0.55, 0.28, 0.05, 0.0}

% 10. Mathematik
\usepackage{amsmath}                % Verbesserter Mathematik Satz
\usepackage{amssymb}                % Zahlenmenngen in der Mathematik
\usepackage{trfsigns}               % Zeichen für Transformationen
\usepackage{mathptmx}               % Times New Roman für Mathematik

% 11. Codesegmente
\usepackage{listings}
\lstdefinelanguage{kmake}{
  keywords={ifeq, else, endif},
  morekeywords={},
  otherkeywords={},
  sensitive=true,
  comment=[l]{\#},
  string=[s]{"}{"},
  showspaces=false,
}
\lstloadlanguages{C, kmake}          % benötigte Sprachen laden
\lstset{                            % Einstellungen für C
  language=C,
  basicstyle=\scriptsize\ttfamily,
  commentstyle=\color{darkgreen},
  keywordstyle=\bfseries\color{darkblue},
  stringstyle=\color{darkred},
%  showspaces=false,
  columns=fixed,
  numbers=left,
%  frame=none,
  numberstyle=\tiny,
  breaklines=true,
%  captionpos=b,
  showstringspaces=false,
% xleftmargin=1cm,
  tabsize=4,
  backgroundcolor=\color{lstbggray},
  numberbychapter=false
}

\let\verbatimorg\verbatim
\renewcommand\verbatim{\small\verbatimorg}


% 12. Metadaten
\title{EEDURO User Manual}
\author{Martin Züger, Stefan Landis}
\date{\today}

% 13. PDF-Einstellungen
\usepackage[
  pdftitle={Build your own EEDURO Delta},
  pdfauthor={Martin Zueger, Stefan Landis},
  pdfcreator={Kile (http://kile.sourceforge.net)},
  pdfsubject={User Manual},
  pdfkeywords={NTB, EEROS, EEDURO, Delta, Robot},
  pdfpagemode=UseOutlines,          % Inhaltsverzeichnis anzeigen beim öffnen
  pdfdisplaydoctitle=true           % Dokumenttitel statt Dateiname anzeigen
  ]{hyperref}
\hypersetup{
  colorlinks=true,
  linkcolor=LinkColor,              % Farbe für Links
  citecolor=LinkColor,              % Farbe für Literaturreferenzen
  filecolor=LinkColor,              % 
  menucolor=LinkColor,              % 
  pagecolor=LinkColor,              % Farbe für Links auf Seitenzahlen
  urlcolor=LinkColor}               % Farbe für URLs

% 14. Eigene Befehle
% 14.1. Text überstreichen
\newcommand{\OL}[1]{$\overline{\text{#1}}$}
